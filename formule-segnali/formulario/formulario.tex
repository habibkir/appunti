\documentclass{article}
\usepackage{steinmetz} % \phase per la fase di un complesso
\usepackage{amsmath} %da abusare
\usepackage{amsfonts} %appartiene a \mathbb{R}
\usepackage{mathrsfs} %mathcal{F} = simbolo di trasformata

\begin{document}

Formule della serie
\begin{itemize}
\item Sintesi \begin{equation*}
  x(t) = \sum_{k=-\infty}^{+\infty} X_k e^{j2\pi kf_0 t}
\end{equation*}
\item Analisi \begin{equation*}
  X_k = \frac{1}{T_0}\int_{-\frac{T_0}{2}}^{+\frac{T_0}{2}} x(t) e^{-j2\pi kf_0 t} dt
\end{equation*}
\item Forma rettangolare
\item Forma polare
\end{itemize}

Criterii di dirichlet (non so se dovrebbero stare in un formulario)
\begin{itemize}
\item $x(t)$ è assolutamente integrabile nel periodo, vale a dire \begin{equation*}
  \int_{\frac{T_0}{2}}^{\frac{T_0}{2}} \lvert x(t) \rvert < \infty
\end{equation*}
\item $x(t)$ è continua o presenta un numero finito di discontiniutà di prima specie (salto) nel periodo
\item $x(t)$ è derivabile ripsetto al tempo nel periodo, escluso al più un numero finito di punti dove esistono finite le derivate destra e sinistra
\end{itemize}
Se sono soddisfatti i criterii la serie di fourier (che è una sommatoria infinita (definizione di serie)) converge al valore della funzione (dove questa è continua) e alla media tra i limiti destro e sinistro della funzione dove questa è discontinua di prima specie (salto)

Casi particolari (quelli con $X_k$ e $X_{-k}$)
\begin{itemize}
\item x(t) reale \begin{equation*}
  X_{-k} = X^*_k \text{ (simmetria Hermitiana)}
\end{equation*}
\item x(t) pari \begin{equation*}
  X_{-k} = X_k 
\end{equation*}
\item x(t) dispari \begin{equation*}
  X_{-k} = -X_k
\end{equation*}
\item x(t) pari e reale \begin{align*}
  &X_{-k} = X_k\\
  &X_k \in \mathbb{R}
\end{align*}
\item x(t) dispari e reale \begin{align*}
  &X_{-k} = -X_k \\
  &X_k\text{ puramente immaginario}
\end{align*}
\end{itemize}
Per la trasformata di Fourier (valgono tutte le simmetrie della serie per tutti gli stessi casi (credo))
\begin{itemize}
\item Analisi \begin{equation*}
  X(f)=\mathcal{F}\{x(t)\}=\int_{-\infty}^{\infty} x(t) e^{-j2\pi ft} dt
\end{equation*}
\item Sintesi \begin{equation*}
  x(t)=\int_{-\infty}^{\infty} X(f) e^{j2\pi ft} df
\end{equation*}
\end{itemize}

Trasformate di funzioni particolari
\begin{itemize}
\item rect \begin{equation*}
  \mathcal{F}\{rect(\frac{t}{B})\} = B \emph{sinc}(tB)
\end{equation*}
\item tri
\item esponenziale monolatera \begin{equation*}
  \mathcal{F}\{e^{\frac{-t}{T}}u(t)\} = \frac{T}{1 + j2\pi fT}
\end{equation*}
\item sinc
\end{itemize}

Teoremi trasformata
\begin{itemize}
\item Teorema della derivazione
\item Teorema dell'integrazione
\item Teorema del ritardo
\item Teorema della modulazione %per la radio
\item Teorema della coseno modulazione %per la radio, doppio
\item Teorema del cambiamento di scala %dimostrazione scientifica delle vocine accellerate
\item Teorema della dualità %chi lo vuole un flashback al Tardella?
\item Teorema della convoluzione
\item Teorema del prodotto
\end{itemize}

Formule per variabili aleatorie : Indici
\begin{itemize}
\item Una variabile \begin{itemize}
\item Media
\item Potenza
\item Varianza
\end{itemize}
\item Più variabili \begin{itemize}
  \item Correlazione
  \item Covarianza
  \item Indice di covarianza
\end{itemize}
\end{itemize}

Sistemi LTI
facciamo che si rappresenta il sistema come $\mathcal {T}$

\end{document}
