\documentclass{article}
\usepackage{amsmath}
\usepackage{amsfonts}
\usepackage{amssymb} % ç\circledast
\usepackage{mathrsfs} % \mathcal{F}

\begin{document}
Basi di numeri complessi

Cose che usa per seni e coseni

Teoremi Abusati

\begin{itemize}
  \item Derivazione \begin{equation*}
    \mathcal{F}\{\frac{dx(t)}{dt}\} = j2\pi f\mathcal{F}\{x(t)\}
  \end{equation*}
  
  \item Ritardo \begin{equation*}
    \mathcal{F}\{x(t-t_0)\} = e^{-j2\pi ft_0} \mathcal{F}\{x(t)\}
  \end{equation*}
\end{itemize}

Trasformate Importanti
\begin{itemize}
  \item Rect \begin{equation*}
    \mathcal{F}\{rect(\frac{t}{\tau }\} = \tau sinc(f\tau )
  \end{equation*}
  
  \item Sinc \begin{equation*}
    \mathcal{F}\{sinc(t\tau )\} = \frac{1}{\tau } rect(\frac{f}{\tau }
    %O qualunque cosa funzioni col teorema del cambiamento di scala
  \end{equation*}
  
  \item Tri \begin{equation*}
    \mathcal{F}\{tri(t)\} = sinc^2(f)
  \end{equation*}
\end{itemize}

Potenze
\begin{itemize}
  \item Potenza istantanea di un segnale $x(t)$ \begin{equation*}
    x^2(t)
  \end{equation*} %minchia che spreco di notazione
    
  \item Potenza media di un segnale periodico $x(t)$ di periodo $T_0$\begin{equation*}
    \lim_{T_0 \to \infty} \frac{1}{T_0} \int_{- \frac{T_0}{2}}^{\frac{T_0}{2}} x^2(t)
  \end{equation*}
  
  \item Potenza media di un segnale generico a potenza finita \begin{equation*}
    \lim_{T_0 \to \infty} \frac{1}{T_0} \int_{- \frac{T_0}{2}}^{\frac{T_0}{2}} x^2(t)
  \end{equation*}
    
  \item \textbf{Potenza di un segnale armonico} di ampiezza $A$ \begin{equation*}
    \frac{A^2}{2} \text{ non ci interessa la fase o la frequenza, solo l'ampiezza}
  \end{equation*}
\end{itemize}

%Cubo di sto materiale con questa densità, l'integrale della densità in sto cubo fa il peso
%Potenza "lineare" se hai una o più medie nulle e bla bla incorrelate

Merdate con integrali e funzioni pari/dispari
\begin{itemize}
  \item Se $x(t)$ pari (ad esempio il coseno) \begin{equation*}
    \int_{-b}^{-a} x(t) dt + \int_{a}^{b} x(t) dt = 2\int_{a}^{b} x(t) dt
  \end{equation*}

  \item Col caso particolare \begin{align*}
    &\int_{-b}^{0} x(t) dt + \int_{0}^{b} x(t) dt = 2\int_{a}^{b} x(t) dt \text{ che possiamo scrivere come}\\
    &\int_{-b}^{b} x(t) dt = 2\int_{a}^{b} x(t) dt
  \end{align*}

  \item Se $x(t)$ dispari (ad esempio il seno) \begin{equation*}
    \int_{-b}^{-a} x(t) dt + \int_{a}^{b} x(t) dt = 0
  \end{equation*}

  \item Col caso particolare \begin{align*}
    &\int_{-b}^{0} x(t) dt + \int_{0}^{b} x(t) dt = 0 \text{ che possiamo scrivere come}\\
    &\int_{-b}^{b} x(t) dt = 0
  \end{align*}
\end{itemize}

Altre formule abusate negli LTI
\begin{itemize}
  \item Risposta impulsiva
  \item Risposta in frequenza
  \item Onda elementare che passa per un LTI
  \item Densità spettrale di potenza / $S_{XX}$ di un segnale che passa per un LTI
\end{itemize}

\end{document}
