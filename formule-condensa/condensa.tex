\documentclass{article}
\usepackage{amsmath}
\usepackage{amsfonts}
\usepackage{amssymb} % ç\circledast
\usepackage{mathrsfs} % \mathcal{F}
%per chi fosse interessato a "ma sto coglione la sa usare una math mode?"
%sto editor non segna la sintassi di un cazzo in math mode

\begin{document}

Basi di numeri complessi

Cose che usa per seni e coseni

Potenze
\begin{itemize}
\item Potenza istantanea di un segnale $x(t)$ \begin{equation*}
  x^2(t)
\end{equation*} %minchia che spreco di notazione
  
\item Potenza media di un segnale periodico $x(t)$ di periodo $T_0$\begin{equation*}
  \lim_{T_0 \to \infty} \frac{1}{T_0} \int_{- \frac{T_0}{2}}^{\frac{T_0}{2}} x^2(t)
\end{equation*}

\item Potenza media di un segnale generico a potenza finita \begin{equation*}
  \lim_{T_0 \to \infty} \frac{1}{T_0} \int_{- \frac{T_0}{2}}^{\frac{T_0}{2}} x^2(t)
\end{equation*}
  
\item \textbf{Potenza di un segnale armonico} di ampiezza $A$( \begin{equation*}
  \frac{A^2}{2} \text{ non ci interessa la fase o la frequenza, solo l'ampiezza}
\end{equation*}
\end{itemize}

%Cubo di sto materiale con questa densità, l'integrale della densità in sto cubo fa il peso
%Potenza "lineare" se hai una o più medie nulle e bla bla incorrelate

Merdate con integrali e funzioni pari/dispari
\begin{itemize}
\item [insert eldricht truth here]
\end{itemize}

Altre formule abusate negli LTI
\begin{itemize}
\item Risposta impulsiva
\item Risposta in frequenza
\item Onda elementare che passa per un LTI
\item Densità spettrale di potenza / $S_{XX}$ di un segnale che passa per un LTI
\end{itemize}

\end{document}
