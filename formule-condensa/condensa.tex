\documentclass{article}
\usepackage{amsmath}
\usepackage{amsfonts}
\usepackage{amssymb} % ç\circledast
\usepackage{mathrfsf} % \mathcal{F}

\begin{document}

Basi di numeri complessi

Cose che usa per seni e coseni

Potenze
\begin{itemize}
\end{itemize}

%Cubo di sto materiale con questa densità, l'integrale della densità in sto cubo fa il peso
%Potenza "lineare" se l'aleatorio ha densità nulla

Merdate con integrali e funzioni pari/dispari

\begin{itemize}
\end{itemize}

Altre formule abusate negli LTI
\begin{itemize}
\item Risposta impulsiva
\item Risposta in frequenza
\item Onda elementare che passa per un LTI
\item Densità spettrale di potenza / S_{XX} di un segnale che passa per un LTIini
\end{itemize}

  

\end{document}
