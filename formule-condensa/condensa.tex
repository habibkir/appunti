\documentclass{article}
\usepackage{amsmath}
\usepackage{amsfonts}
\usepackage{amssymb} % ç\circledast
\usepackage{mathrsfs} % \mathcal{F}

\begin{document}
(L'ho riguradato un paio di volte, sono \emph{abbastanza} sicuro di non averlo scazzato)

Basi di numeri complessi
\begin{itemize}
\item Devo ancora aggiungere la roba qui, scusate
\end{itemize}

Teoremi Abusati (per tutti i seguenti $X(f) = \mathcal{F}\{x(t)\}$)
\begin{itemize}
  \item Derivazione \begin{equation*}
    \mathcal{F}\{\frac{dx(t)}{dt}\} = j2\pi f X(f)
  \end{equation*}
  
  \item Ritardo \begin{equation*}
    \mathcal{F}\{x(t-t_0)\} = e^{-j2\pi ft_0} X(f)
  \end{equation*}

  \item Coseno modulazione \begin{equation*}
    \mathcal{F}\{x(t) cos(2\pi f_0 t)\} = \frac{1}{2}(X(f)\rvert _{f=f-f_0} + X(f)\rvert _{f=f+f_0})
  \end{equation*}
\end{itemize}

Trasformate Importanti
\begin{itemize}
  \item Rect \begin{equation*}
    \mathcal{F}\{rect(\frac{t}{B}\} = B sinc(fB)
  \end{equation*}
  
  \item Sinc (dualità del coso prima) \begin{equation*}
    \mathcal{F}\{B sinc(Bt)\} = rect(\frac{f}{B})
    %O qualunque cosa funzioni col teorema del cambiamento di scala
  \end{equation*}
  
  \item Tri (coso del rect $\circledast$ coso del rect) \begin{equation*}
    \mathcal{F}\{tri(t)\} = sinc^2(f)
  \end{equation*}
\end{itemize}

Potenze
\begin{itemize}
  \item Potenza istantanea di un segnale $x(t)$ \begin{equation*}
    x^2(t)
  \end{equation*} %minchia che spreco di notazione
    
  \item Potenza media di un segnale periodico $x(t)$ di periodo $T_0$\begin{equation*}
    \frac{1}{T_0} \int_{- \frac{T_0}{2}}^{\frac{T_0}{2}} x^2(t)
  \end{equation*}
  
  \item Potenza media di un segnale generico a potenza finita \begin{equation*}
    \lim_{T_0 \to \infty} \frac{1}{T_0} \int_{- \frac{T_0}{2}}^{\frac{T_0}{2}} x^2(t)
  \end{equation*}

  \item Potenza di un sengnale data la densità spettrale di potenza \begin{equation*}
    P_x = \int_{-\infty}^{\infty} S_{XX}(f) df
  \end{equation*}
    
  \item \textbf{Potenza di un segnale armonico} di ampiezza $A$ \begin{equation*}
    \frac{A^2}{2} \text{ non ci interessa la fase o la frequenza, solo l'ampiezza}
  \end{equation*}
\end{itemize}

%Cubo di sto materiale con questa densità, l'integrale della densità in sto cubo fa il peso
%Potenza "lineare" se hai una o più medie nulle e bla bla incorrelate

Merdate con integrali e funzioni pari/dispari
\begin{itemize}
  \item Se $x(t)$ pari (ad esempio il coseno) \begin{equation*}
    \int_{-b}^{-a} x(t) dt + \int_{a}^{b} x(t) dt = 2\int_{a}^{b} x(t) dt
  \end{equation*}

  \item Col caso particolare \begin{align*}
    &\int_{-b}^{0} x(t) dt + \int_{0}^{b} x(t) dt = 2\int_{a}^{b} x(t) dt \text{ che possiamo scrivere come}\\
    &\int_{-b}^{b} x(t) dt = 2\int_{0}^{b} x(t) dt
  \end{align*}

  \item Se $x(t)$ dispari (ad esempio il seno) \begin{equation*}
    \int_{-b}^{-a} x(t) dt + \int_{a}^{b} x(t) dt = 0
  \end{equation*}

  \item Col caso particolare \begin{align*}
    &\int_{-b}^{0} x(t) dt + \int_{0}^{b} x(t) dt = 0 \text{ che possiamo scrivere come}\\
    &\int_{-b}^{b} x(t) dt = 0
  \end{align*}
\end{itemize}

Altre formule abusate negli LTI, scriviamo il sistema come $y(t) = \mathcal{T}[x(t)]$
\begin{itemize}
  \item Risposta impulsiva \begin{align*}
    &h(t) = \mathcal{T}[\delta (t)]\\
    &y(t) = h(t) \circledast x(t)
  \end{align*}
  
  \item Risposta in frequenza \begin{align*}
    &H(f) = \mathcal{F}\{h(t)\}\\
    &Y(f) = X(f) H(f)\\
    &H(f) = \frac{Y(f)}{X(f)}\\
  \end{align*}
    
  \item Onda elementare che passa per un LTI \begin{align*}
    &x(t) = A cos(2\pi f_0t + \phi_0) \Longrightarrow y(t) = \lvert H(f_0) \rvert A cos(2\pi f_0t + \phi_0 + \angle H(f_0))\\
    &x(t) = A sin(2\pi f_0t + \phi_0) \Longrightarrow y(t) = \lvert H(f_0) \rvert A sin(2\pi f_0t + \phi_0 + \angle H(f_0))\\
  \end{align*}
    
  \item Densità spettrale di potenza ($S_{XX}$) di un segnale che passa per un LTI \begin{equation*}
    S_{yy} = S_{XX} \lvert H(f) \rvert ^2
  \end {equation*}
\end{itemize}

\end{document}
