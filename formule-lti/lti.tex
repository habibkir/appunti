\documentclass{article}
\usepackage{amsmath}
\usepackage{amsfonts}
\usepackage{amssymb} %\circledast (circled asterisk) per la convoluzione
\usepackage{mathrsfs} %\mathcal{F} per fourier

\begin{document}
diciamo che $y(t) = T[x(t)]$
\begin{itemize}

\item Risposta impulsiva \begin{equation*}
  h(t) = T[\delta (t)]
\end{equation*}

\item Risposta in frequenza \begin{equation*}
  H(f) = \mathcal{F}\{h(t)\} = \int_{-\infty}^{\infty} h(t) e^{-j2\pi f_0 t} dt
\end{equation*}

\item Uscita del sistema \begin{align*}
  y(t) &= h(t) \circledast x(t)\\
  Y(f) &= H(f) \times X(f)
\end{align*}

\item Risposta nel tempo sistemi in serie\begin{align*}
  y(t) &= (x(t) \circledast h_1(t)) \circledast h_2(t)\\
       &= x(t) \circledast (h_1(t) \circledast h_2(t))\\
  h_{eq}(t)&= h_1(t) \circledast h_2(t)
\end{align*}

\item Risposta in frequenza sistemi in serie \begin{align*}
  Y(f) &= (X(f) * H_1(f)) * H_2(f)\\
       &= X(f) * (H_1(f) * H_2(f))\\
  H_{eq}(f) &= H_1(f) * H_2(f)
\end{align*}

\item Sistemi in parallelo \begin{align*}
  h_{eq}(t) &= h_1(t) + h_2(t)\\
  H_{eq}(f) &= H_1(f) * H_2(f)
\end{align*}

\item Risposta per input armonico \begin{align*}
  x(t) &= A cos(2\pi f_0 t + \phi _0)\\
  y(t) &= \lvert H(f_0) \rvert A cos(2\pi f_0 t + \phi _0 + \angle H(f_0))
\end{align*}

\end{itemize}

\end{document}


